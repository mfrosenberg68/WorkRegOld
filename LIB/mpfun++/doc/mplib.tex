\documentstyle[11pt,fullpage]{article}

\def\Fninety{Fortran~90}
\def\CC{C\raise.22ex\hbox{{\footnotesize +}}\raise.22ex\hbox{\footnotesize +}}
\def\mp{\verb|mp|}

\title{A \CC-based Multiprecision System}

\author{Siddhartha Chatterjee
        \thanks{Research Institute for Advanced Computer Science, 
	        Mail Stop T27A-1, 
		NASA Ames Research Center, 
		Moffett Field, CA 94035-1000 (sc@riacs.edu).
		This work was supported by the NAS Systems Division
		via Contract NAS 2-13721 between NASA and the
		Universities Space Research Association (USRA).}
	}

\date{September 1994}

\begin{document}

\maketitle

\begin{abstract}
To be written.
\end{abstract}

\section{Introduction}

This document describes a \CC\ interface to David Bailey's
Fortran-based multiprecision
package~\cite{bailey:90,bailey:93,bailey:94}.
Two events prompted the writing of the \CC\ interface.
First,
David rewrote his original Fortran-77 based system to take advantages
of the derived data types and module facilities of \Fninety.
Second,
he apprised me that there were a large number of potential users of
his system,
were it to be made available in C/\CC.
It was natural to use classes and operator overloading in \CC\ to
achieve the same effect as David achieved with \Fninety\ modules.

The \Fninety\ multiprecision library consists of two
files---mpmod90.f, which defines the module interfaces (6557 lines);
and mpfun90.f, which provides the actual computation routines (8732
lines).
From the beginning,
I decided not to rewrite the computation routines,
but to use external linkage to call them from the \CC\ code.
This restricted the \CC\ classes to have the same memory layout as the
\Fninety\ derived types,
but I deemed this preferable to rewriting several thousand lines of
intricate and sophisticated numerical computations.
Further,
the \Fninety\ module interfaces result in a explosion of routines
defining mixed-mode operations between the three derived
multiprecision types and the basic \Fninety\ data types.
I wanted to avoid such an explosion,
and therefore decided to use automatic conversion and ambiguity
resolution in \CC\ to achieve the same effect.
This is also more in line with \CC\ philosophy
(see Stroustrup's discussion of this
issue~\cite[p.~80]{stroustrup:94}).

The remainder of this document serves two purposes:
it is the user guide for the library,
and it documents and explains design decisions.

The library has been tested on the following platforms: 
Sun-4 with CC version SC2.0.1,
SGI with CC version 2.2,
IBM-RS6000/590 with xlC,
and IBM-RS6000/590 with g++ version 2.6.0.
{\bf The library does not compile with g++ versions prior to 2.6.0.}

\section{Classes and operations}

The library exports three classes: 
\verb|mp_integer|, \verb|mp_real|, and \verb|mp_complex|.
Call the procedure \verb|mp_init()| before performing any
multiprecision operations in the program.
This procedure sets parameters in the mpfun library and precomputes
$\pi$, $\log 2$, and $\log 10$,
which are needed in a number of transcendental function routines.

As \CC\ does not have a built-in complex data type,
the file \verb|complex.H| defines two skeletal classes \verb|FComplex|
(single-precision complex) and \verb|DComplex| (double-precision
complex).
These are also the names of the corresponding classes in the Rogue
Wave classes,
and some day I will get around to fixing the code to use those classes
if they are available.

\subsection{Constructors and destructors}

Each of the three \mp\ classes has constructors from the types
\verb|void|, \verb|int|, \verb|float|, \verb|double|, and
\verb|char*|.
String constants for representing \mp\ variables follow the
formatting conventions of Fortran.
Thus,
one can say
\begin{verbatim}
    mp_integer ia = -1.0;
    mp_real p = 5;
    mp_complex u[100];
    mp_real x = "1.234567890 1234567890 1234567890 D-100";
\end{verbatim}
In addition,
an \verb|mp_real| can be constructed from an \verb|mp_integer|,
and an \verb|mp_complex| can be constructed from an \verb|mp_integer|
or an \verb|mp_real|.

The \mp\ classes have no explicit copy constructors,
since the default action of bitwise copying is the correct one.
One can say
\begin{verbatim}
    mp_real ee = exp(mp_real(1.0));
\end{verbatim}
They also lack explicit destructors for similar reasons.

\subsection{Assignment}

Each of the \mp\ classes has assignment operators whose argument
can come from any of the \mp\ classes or from the base types
\verb|int|, \verb|float|, \verb|double|, or \verb|char*|.
The current implementation does not provide \verb|operator int()|,
\verb|operator float()|,
\verb|operator double()|,
and \verb|operator char *()|
that would allow the transparent assignment of \mp\ variables to
the base types,
as this causes ambiguity problems in mixed-mode operations.
This problem will be fixed in a later version of the library.
As an intermediate fix,
use the friend functions \verb|INT|, \verb|REAL|, and \verb|DBLE|.
Thus,
\begin{verbatim}
    mp_integer id, k;
    int p, nd4;

    id = power(mp_real(10), nd4)/2;
    p = INT(k);
\end{verbatim}

\subsection{Operator overloading}

The following operators are overloaded for the \mp\ classes:
\verb|+|,
\verb|-| (both unary and binary),
\verb|*|,
\verb|/|,
%\verb|%|,
\verb|+=|,
\verb|-=|,
\verb|*=|,
\verb|/=|,
\verb|==|,
\verb|!=|,
\verb|<=|,
\verb|>=|,
\verb|<|,
\verb|>|,
\verb|>>|,
and \verb|<<|.
The binary arithmetic operations have prototypes of the form
\begin{verbatim}
    friend const T operator+(const T&, const T&);
\end{verbatim}
where \verb|T| is one of \verb|mp_integer|, \verb|mp_real|, or
\verb|mp_complex|.
The argument matching facility of \CC\ handles mixed-mode operations.
Thus,
the following statements are all legal.
\begin{verbatim}
    int i, j;
    double a, b;
    mp_integer mi, mj;
    mp_real ma, mb;
    mp_complex mc;

    mc += i;
    ma = (mb*mi)/b;
    j = (a*mb == ma-mi) ? INT(mi) : INT(ma);
    cout << mi << ma << mc;
\end{verbatim}

The overloading of the stream I/O operations addresses one of David's
complaints about the \Fninety\ version of his library,
in that special input and output routines need not be called to read
and write \mp\ variables.

\subsection{Other functions}

Each of the \mp\ classes has some other functions defined for it,
such as power, 
logarithms, 
trigonometric functions, 
and transcendental functions.
Refer to the file \verb|mpmod.H| for a complete listing.

\begin{table}
\begin{center}
\begin{tabular}{|c|c|c|c|c|} \hline
Name & Description & Type & Dynamic? & Default \\ \hline
\verb|mpipl| & Maximum and initial precision, in digits & int & no &
100 \\
\verb|mpiou| & Initial output precision, in digits & int & no & 56 \\
\verb|mpiep| & $\log_{10}$ of initial MP epsilon & int & no &
10$-$\verb|mpipl| \\ 
\hline
\verb|mpwds| & Maximum and initial precision, in words & int & no &
$\frac{\verb|mpipl|}{7.227472}+1$ \\
\verb|mpoud| & Current output precision, in digits & int & yes &
\verb|mpiou| \\
\verb|mpeps| & Current MP epsilon value & \verb|mp_real| & yes &
$10^{\verb|mpiep|}$ \\
\verb|mpl02| & $\log 2$ & \verb|mp_real| & no & $\log 2$ \\
\verb|mpl10| & $\log 10$ & \verb|mp_real| & no & $\log 10$ \\
\verb|mppic| & $\pi$ & \verb|mp_real| & no & $\pi$ \\ 
\hline
\verb|idb| & MPFUN debug level & \verb|int| & yes & 0 \\
\verb|mcr| & Crossover point for advanced routines & \verb|int| & yes
& 7 \\ 
\hline
\end{tabular}
\end{center}
\caption{Tunable parameters for the MP package.}
\label{tab:parameters}
\end{table}

\section{Tunable parameters}

The \mp\ library and the file \verb|mpfun.f| define a number of
global variables,
as shown in \ref{tab:parameters}.
The column marked ``Dynamic?'' indicates whether a parameter can be
changed dynamically during program execution.
The first group of three parameters can be changed using the
\verb|-D<name>=<value>| option to make or the \CC\ compiler,
where \verb|<name>| is one of \verb|MPIPL|, \verb|MPIOU|, or
\verb|MPIEP|.
The second group of variables defines some often-needed mathematical
constants,
and also some parameters related to machine precision.
The final group of parameters are defined in \verb|mpfun.f| and must
be read and set in a rather arcane way if you are linking to the
Fortran~77 version.
The procedure
\begin{verbatim}
    void MPINQP(const char* s, int& v, const int& n);
\end{verbatim}
reads the variable named in \verb|s| into \verb|v|,
while the procedure
\begin{verbatim}
    void MPSETP(const char* s, const int& v, const int& n);
\end{verbatim}
sets the value of the variable named in \verb|s| to \verb|v|.
If you are using the \Fninety\ version of the library
(\verb|mpfun90.f| rather than \verb|mpfun.f|),
these parameters are called \verb|mpidb| and \verb|mpmcr| and can be
read and written as usual.

The files \verb|mpfun.f| and \verb|mpfun90.f| define some more
settable parameters,
that you should only play with if you are tinkering with the numerical
algorithms.
Refer to~\cite{bailey:94} for a complete list of these parameters.

\section{Design decisions}

I had several major design goals for the \CC\ interface to the \mp\
library:
\begin{enumerate}
\item
Make the transition from the standard arithmetic types to the \mp\
types as smooth as possible for the user.
This meant a full complement of constructors,
initializers,
the assignment operation,
and other arithmetic and logical operations.
I wanted to make mixed-mode operations as transparent as possible,
with the user being required to perform explicit conversions chiefly
for the sake of controlling precision.

\item
Make the interface work under a variety of \CC\ compilers,
including true compilers such as \verb|gcc| and cfront-based
translators such as \verb|CC|.
\end{enumerate}

From an implementation standpoint,
I wanted to avoid the explosion of routines as had happened in David's
\Fninety\ version of the library.
(In his implementation,
the numerical routines comprise 8732 lines of code,
while the interface routines comprise 6557 lines of code.)
He explicitly coded a routine for every possible combination of
argument types.
In his case,
he had no other choice,
since the mechanisms for supporting generic functions in \Fninety\ are
quite basic and limited.
\CC,
on the other hand,
has a sophisticated system of argument matching and ambiguity
control~\cite{ellis:stroustrup:90} to match up definitions and calls
of overloaded operators.
I wanted to exploit this facility to the fullest and keep the size of
the library small.
While the explicit approach can gain some advantage by combining the
argument conversions with the arithmetic,
this gain is minimal in this case as the arithmetic routines usually
dominate the computation cost.

\subsection{Mixed mode operations}
However,
interactions among these goals make them unattainable without some
compromise.
Consider the interaction between assignment and addition.
Since the basic arithmetic types are not \CC\ classes and are treated
specially,
the only transparent way to assign an \verb|mp_real r| to an \verb|int
i| is to define \verb|operator int()| as a member function for class
\verb|mp_real|.
Then the assignment \verb|i = r;| is just shorthand for \verb|i =
r.operator int();|.
Now suppose that we have defined
\begin{verbatim}
    mp_real(const int);
\end{verbatim}
and
\begin{verbatim}
    friend const mp_real operator+(const mp_real&, const mp_real&);
\end{verbatim}
for class \verb|mp_real|.
Then the statement \verb|i = r+i;| is ambiguous;
for,
according to the argument matching rules of the language,
either of the interpretations \verb|i = operator+(r, mp_real(i));|
and \verb|i = r.operator() + i;| is equally good.
Note that while I have illustrated this problem with an \verb|int| and
an \verb|mp_real|,
this same problem surfaces in mixed-mode operations between \mp\
classes if every \mp\ class has a constructor from every other \mp\ 
class.

One obvious solution is to go back to the explicit approach,
but I was lazy and loath to do that just yet.
Instead,
I wanted to pursue other approaches to limit code growth.
My solution was to provide a sparse set of contructors and the full
cross-product of assignment operators,
and not to provide the conversion operators to the basic types.
This limits the code explosion and solves the problem of unambiguous
mixed-mode operations for the \mp\ types,
but does not allow transparent assignment of \mp\ variables to
integers and other basic types.
Since the overloaded assignment operator must be a nonstatic member
function of a class,
there is no way of redefining assignment for the basic types.
\Fninety\ does not have this particular problem;
it allows redefinition of assignment for arbitrary types of the
assignment target and the assigned value.

There was one further problem;
even with these changes to remove ambiguity,
\verb|gcc| version~2.5.8 would not correctly compile the \mp\ library.
This appears to have been a compiler bug that has been fixed in
\verb|gcc| version~2.6.0.
The current version of the library is the one described in the
previous paragraph.

Much as I hate to admit it,
I feel that,
with the current compilers available to users,
the only robust way to implement this library is to bypass the
argument matching mechanism of \CC\ and revert to the explicit
approach that David employed.
This is a disappointing realization but perhaps a realistic compromise
between elegance and reliability.

\subsection{External linkage with Fortran code}

As the \CC\ wrappers call Fortran subroutines to perform the numeric
computations,
we need an external linkage mechanism to correctly link the \CC\ and
Fortran programs.
Fortran passes parameters by reference,
while \CC\ by default passes parameters by value/result.
However,
\CC\ allows the user to specify the passing of parameters by
reference,
by using the reference type.
Finally,
the use of the keyword \verb|const| allows the semantics of the
\verb|INTENT| attribute in \Fninety.

\CC\ has special syntax to specify external linkage.
However,
C is the only language with which all \CC\ compilers support external
linkage.
Of the compilers I used,
only IBM's \verb|xlC| supports the \verb|extern "FORTRAN"| linkage.
For all other compilers,
I had to mangle the names of the Fortran routines and use the
\verb|extern "C"| linkage.
Thus,
on the IBM RS6000/590,
the Fortran subroutine with the declaration
\begin{verbatim}
       SUBROUTINE MPCPWR(L, A, N, B)
      DIMENSION A(2*L), B(2*L)
\end{verbatim}
would be declared
\begin{verbatim}
    extern "FORTRAN" mpcpwr(const int&, const float*, const int&, float*);
\end{verbatim}
using the \verb|xlC| compiler,
while it would be declared
\begin{verbatim}
    extern "C" mpcpwr_(const int&, const float*, const int&, float*);
\end{verbatim}
using \verb|g++|.

One final nonportability concerns the Fortran libraries that need to
be linked in.
Depending on the hardware platform and compiler,
the libraries required might be libxlf90.a and libxlf.a;
or libI77.a, libF77.a, and libisam.a;
or even libM77.a and libF77.a.

\bibliographystyle{plain}
\bibliography{mplib}

\end{document}
